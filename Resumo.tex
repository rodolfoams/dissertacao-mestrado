No decorrer das últimas décadas, uma quantidade de dados de usuários cada vez
maior vem sendo enviada para ambientes não controlados pelos mesmos. Em alguns
casos esses dados são enviados com o objetivo de tornar esses dados públicos,
mas na grande maioria das vezes há a necessidade de manter esses dados seguros e
privados, ou autorizar o seu acesso apenas em usos bem específicos. Considerando
o caso onde os dados devem ser mantidos privados, entidades devem tomar cuidados
especiais para manter a segurança e privacidade de tais dados tanto durante a
transmissão quanto durante o armazenamento e processamento dos mesmos. Com esse
objetivo, vários esforços vêm sendo feitos, inclusive o desenvolvimento de
componentes de \textit{hardware} que provêem ambientes de execução confiável,
\textit{TEEs}, como o \textit{Intel Software Guard Extensions} (\textit{SGX}).
O uso dessa tecnologia, porém, pode ser feito de forma incorreta ou ineficiente,
devido a cuidados não observados durante o desenvolvimento de aplicações.

O trabalho apresentado nessa dissertação aborda os principais desafios
enfrentados no desenvolvimento de aplicações que façam uso de \textit{SGX}, e
propõe boas práticas e um conjunto de ferramentas (\textit{DynSGX}) que ajudam
a fazer melhor uso das capacidades da tecnologia. Tais desafios incluem, mas não
são limitados a, particionamento de aplicações de acordo com o modelo de
programação do SGX, colocação de aplicações em ambientes de computação na nuvem,
e, sobretudo, gerência de memória.

Os estudos apresentados neste trabalho apontam que o mal uso da tecnologia pode
acarretar em uma perda de performance considerável se comparado com
implementações que levam em conta as boas práticas propostas. O conjunto de
ferramentas proposto neste trabalho também mostrou possibilitar a proteção de
código de aplicações em ambientes de computação na nuvem, com uma sobrecarga
desprezível em comparação com o modelo de programação padrão de SGX.