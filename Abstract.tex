During the last few decades, an increasing amount of user data have been sent
to environments not controlled by data owners. In some cases these data are sent with
the objective to turn them public, but in the vast majority of times, these data
need to be kept safe and private, or to be allowed access only in very specific
use cases. Considering the case where data need to be kept private, entities
must take specific measures to maintain the data security and privacy while
transmitting, storing and processing them. With this objective many efforts
have been made, including the specification of hardware components that provide
a trusted execution environment (TEEs), like the Intel Software Guard Extensions
(SGX). The use of this technology , though, can be made in incorrect or
ineffective ways, due to not taking some considerations into account during the
development of applications.

In this work, we approach the main challenges faced in the development of
applications that use SGX, and propose good practices and a toolset (DynSGX)
that help making better use of the capabilities of this technology. Such
challenges include, but are not limited to, application partitioning,
application colocation in cloud computing environments, and memory management.

The studies presented in this work show that the bad use of this technology can
result in a considerable performance loss when compared to
implementations that take into account the good practices proposed. The toolset
proposed in this work also showed to enable protecting application code in cloud
computing environments, having a negligible performance overhead when
compared to the regular SGX programming model.