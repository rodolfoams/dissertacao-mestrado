\chapter{Exemplo de aplicação SGX}
\label{appendix:aplicacao_sgx}

Como discutido na Seção~\ref{sec:sgx_modelo_programacao}, aplicações SGX são
divididas entre parte segura, que executa dentro de enclaves SGX, e parte
insegura, usada para instanciar o enclave da aplicação, executar ações que não
necessitem de garantias de segurança, e intermediar qualquer tipo decomunicação
entre um enclave e o mundo externo a ele, inclusive a execução de \textit
{system calls}. Além disso, precisamos definir uma interface de comunicação
entre as partes segura e insegura (arquivo EDL).

Para exemplificar o desenvolvimento de aplicações SGX, utilizamos a seguir a
aplicação utilizada nos experimentos da Seção~\ref
{sec:abordagem_particionamento_dados}.

\section{Desenvolvimento do arquivo EDL}
\label{sec:appendix_aplicacao_sgx_arquivo_edl}

A primeira coisa a se pensar no desenvolvimento de aplicações SGX é em como
particionar a aplicação entre as partes segura e insegura, e, mais precisamente,
é necessário pensar em como ambas as partes se comunicam entre si. Para isso,
precisamos escrever um arquivo EDL, contendo a declaração das ECalls (declaradas
na seção \textit{trusted} do arquivo EDL), e das OCalls (declaradas na seção
\textit{untrusted} do arquivo EDL).

No caso da aplicação em questão, apenas a função \textit{enclave\_sum} cruza
entre as partes segura e insegura. O conteúdo do arquivo EDL está descrito no
Código Fonte~\ref{lst:arquivo_edl}. No código apresentado, o parâmetro "\textit
{in}", entre colchetes, indica que o \textit{array} é um ponteiro de entrada, ou
seja, será copiado da parte insegura para o enclave antes da execução da função
\textit{enclave\_sum}. O parâmetro "\textit{size=array\_size}" indica que o
tamanho do \textit{array}, em \textit{bytes}, é dado pela variável \textit
{array\_size}. 

\begin{lstlisting}[language=C, label=lst:arquivo_edl, caption={Arquivo EDL
contendo a declaração das funções que cruzam a fronteira entre as partes segura
e insegura da aplicação SGX.}]
    // Arquivo enclave.edl
    enclave{
        trusted{
            public long enclave_sum(
                [in, size=array_size] int *array,
                                   size_t  array_size
            );
        };
        untrusted{}; // Nenhuma função a declarar aqui.
    };
\end{lstlisting}

Após a criação do arquivo EDL, utlizamos a ferramenta \textit{sgx\_edger8er} para
compilar o arquivo EDL, e gerar as interfaces de comunicação entre as partes
segura e insegura propriamente ditas. Executamos, portanto, os comandos
\lstinline[breaklines=true,keywordstyle=\ttfamily]!sgx_edger8r --trusted enclave.edl --search-path ${SGXSDK_PATH}/include!
e \lstinline[breaklines=true,keywordstyle=\ttfamily]!sgx_edger8r --untrusted enclave.edl --search-path ${SGXSDK_PATH}/include!,
para gerar os arquivos de interface a serem utilizados, respectivamente, pelas
partes segura e insegura da aplicação.

Mais informações sobre a sintaxe dos arquivos EDL pode ser encontrada no
documento de referência para desenvolvedores~\cite{sgxdeveloperreference}.

\section{Desenvolvimento do enclave}
\label{sec:appendix_aplicacao_sgx_codigo_enclave}

O código do enclave contém a implementação de todas as ECalls declaradas no
arquivo EDL, adicionando-se o código de quaisquer funções internas necessárias.
No caso da aplicação em questão, precisamos escrever apenas a implementação da
função \textit{enclave\_sum}. Tal implementação pode ser encontrada no Código
Fonte~\ref{lst:arquivo_enclave_cpp}.

\newpage
\begin{lstlisting}[language=C, label=lst:arquivo_enclave_cpp, caption={Arquivo
contendo a implementação das funções a serem executadas totalmente dentro do
enclave da aplicação SGX.}]
    // Arquivo enclave.cpp
    #include "stdlib.h"
    #include "enclave_t.h" // Gerado pela ferramenta sgx_edger8er

    long enclave_sum( int *array, size_t array_size) {
        long sum = 0;
        int i;
        for (i=0; i<array_size/sizeof(int); ++i){
            sum += array[i];
        }
        return sum;
    }
\end{lstlisting}

O enclave deve, então, ser compilado com o compilador \textit{g++}, de modo a
gerar uma biblioteca dinâmica (\textit{.so} no caso do Linux, ou \textit{.dll}
no caso do Windows). Por fim, o enclave deve ser assinado usando a ferramenta
\textit{sgx\_signer}, usando o comando
\lstinline[breaklines=true,keywordstyle=\ttfamily]!sgx_signer sign -key priv_key -enclave enclave.so -out enclave.signed.so -config enclave.config.xml!,
onde {\ttfamily priv\_key} é a chave privada do desenvolvedor, {\ttfamily
enclave.so} é a biblioteca dinâmica do enclave produzido pelo \textit
{sgx\_edger8er}, {\ttfamily enclave.signed.so} é o enclave resultante, assinado,
e {\ttfamily enclave.config.xml} é o arquivo de configuração das características
do enclave SGX, conforme exibido no Código Fonte~\ref
{lst:arquivo_enclave_config_xml}.

Entre as configurações do arquivo \textit{enclave.config.xml} que podemos,
destacar, estão o tamanho máximo da \textit{heap} e tamanho máximo da pilha do
enclave. Ambas configurações têm impacto no direto no tempo necessário para a
instanciação de um enclave, pois o SGX aloca todo esse espaço de memória durante
a instanciação do enclave. No caso caso específico da configuração do enclave
utilizada nos experimentos da Seção~\ref{sec:abordagem_particionamento_dados},
a instanciação de um enclave levava cerca de 30 segundos para completar.
Enclaves com tamanhos máximos de \textit{heap} e pilha menores que os usados
nesta configuração levam menos tempo para serem instanciados.

\newpage
\begin{lstlisting}[language=XML, label=lst:arquivo_enclave_config_xml, caption={Arquivo
de configuração do enclave SGX.}]
    // Arquivo enclave.config.xml
    <EnclaveConfiguration>
    <ProdID>0</ProdID>
    <ISVSVN>0</ISVSVN>
    <StackMaxSize>0x8000</StackMaxSize>
    <HeapMaxSize>0x100000000</HeapMaxSize>
    <TCSNum>10</TCSNum>
    <TCSPolicy>1</TCSPolicy>
    <DisableDebug>0</DisableDebug>
    <MiscSelect>0</MiscSelect>
    <MiscMask>0xFFFFFFFF</MiscMask>
</EnclaveConfiguration>
\end{lstlisting}

Mais informações sobre o arquivo de configuração do enclave podem ser
encontradas no documento de referência para desenvolvedores~\cite
{sgxdeveloperreference}.

\section{Desenvolvimento da parte insegura da aplicação}
\label{sec:appendix_aplicacao_sgx_parte_insegura}

A parte insegura de uma aplicação SGX compreende todo o código que lida com
dados não sensíveis, adicionando-se o código necessário para instanciar e se
comunicar com o enclave, e a implementação de quaisquer OCalls que tenham sido
declaradas no arquivo EDL. No caso da aplicação em questão, nenhuma OCall
precisa ser implementada. O código necessário para a parte insegura desta
aplicação, é exibido no Código Fonte~\ref{lst:arquivo_app_cpp}.

O fluxo da aplicação pode ser descrito nos seguintes passos:

\begin{enumerate}
    \item A aplicação instancia o enclave SGX (linhas 13-19);
    \item A aplicação gera um \textit{array} com 1GB, povoado por números
    inteiros aleatórios (linhas 20-22);
    \item A aplicação invoca a função do enclave que calcula a soma dos valores
    contidos em uma partição do \textit{array} (linha 29);
    \item A aplicação soma o resultado obtido do enclave ao total armazenado na
    variável \textit{sum} (linha 30);
    \item A aplicação repete os passos 3 e 4 até que todas as partições do
    \textit{array} tenham sido processadas (linhas 27-31);
    \item A aplicação exibe o resultado da soma de todos os elementos do \textit
    {array} (linha 32);
    \item A aplicação destrói o enclave (linha 33).
\end{enumerate}

\begin{lstlisting}[language=C, label=lst:arquivo_app_cpp, caption={Arquivo
contendo a implementação da parte insegura da aplicação SGX.}]
    // Arquivo app.cpp
    #include "stdlib.h"
    #include "enclave_u.h" // Gerado pela ferramenta sgx_edger8er
    #include "sgx_urts.h" // Necessário para instanciar enclaves

    #define ENCLAVE_NAME "enclave.signed.so"
    #define MB_SIZE 1024*1024
    #define ARRAY_SIZE 4096
    #define CHUNK_SIZE 1

    int main(){

        sgx_enclave_id_t eid;
        sgx_status_t ret = SGX_SUCCESS;
        sgx_launch_token_t launch_token;
        int updated = 0;
        ret = sgx_create_enclave( // Instancia um enclave SGX
                ENCLAVE_NAME, SGX_DEBUG_FLAG, &launch_token,
                &updated, &eid, NULL );
        int total_num_of_elements = ARRAY_SIZE * MB_SIZE / sizeof(int);
        int *array = (int *) calloc(total_num_of_elements,sizeof(int));
        generate_random_numbers(array, total_num_of_elements); // Preenche array com números inteiros aleatórios
        int chunk_num_of_elements = CHUNK_SIZE * MB_SIZE / sizeof(int);
        long sum = 0, partial_sum;
        int i, idx;

        for(i=0; i<total_num_of_elements/chunk_num_of_elements; ++i){
            idx = i*chunk_num_of_elements;
            enclave_sum(eid, &partial_sum, &array[idx], CHUNK_SIZE);
            sum += partial_sum;
        }
        printf("A soma dos elementos do array é: %ld.\n", sum);
        sgx_destroy_enclave(eid);
        return 0;
    }
\end{lstlisting}