\chapter{Conclusão}
\label{chapter:conclusao}

Neste capítulo, apresentamos, na Seção~\ref{sec:conclusao_sumario}, um sumário
do que foi proposto e realizado nesta pesquisa. Além disso, apresentamos algumas
considerações sobre as análises feitas, bem como sobre a ferramenta proposta no
Capítulo~\ref{chapter:dynsgx}. Por fim, na Seção~\ref
{sec:conclusao_limitacoes_trabalhos_futuros}, apresentamos as limitações deste
trabalho, e possíveis trabalhos futuros.

\section{Sumário}
\label{sec:conclusao_sumario}

Nesta pesquisa de mestrado discutimos os principais desafios no desenvolvimento
de aplicações com propriedades de segurança e privacidade de dados, usando a
tecnologia Intel SGX. Para tanto, no Capítulo~\ref{chapter:abordagem}, 
analisamos como várias decisões na implementação e no uso de aplicações SGX
influenciam no seu desempenho geral, e nas garantias de segurança realacionadas
ao tamanho do TCB da aplicação. As características analisadas incluem como
devemos gerenciar a memória de enclaves SGX, como deve ser o particionamento
entre parte segura e parte insegura das aplicações SGX, e como deve ser o
particionamento dos dados a serem processados dentro de enclaves SGX.

As análises apresentadas servem para auxiliar desenvolvedores na tomada de
decisões sobre boas práticas a serem aplicadas na arquitetura de aplicações
seguras, diferenciando-se no particionamento de código e dados, e gerência de
memória de tais aplicações em comparação a aplicações inseguras, onde esses
aspectos não têm tanto impacto.

Este trabalho foi motivado pela crescente popularidade do uso da tecnologia
Intel SGX para desenvolver aplicações que proveem segurança e privacidade de
dados, observada na literatura, e pela falta da definição de padrões no
desenvolvimento de tais aplicações, a não ser as recomendações feitas pela
própria Intel. Em experimentos realizados em nosso ambiente, constatamos que o
mau planejamento de uma aplicação SGX pode acarretar uma grande perda de
desempenho da mesma, chegando a um custo de execução até $534\%$ mais alto,
comparado a aplicações bem planejadas.

Considerando as análises feitas, propomos uma ferramenta, DynSGX, que serve
como uma alternativa ao modelo de programação do Intel SGX, facilitando o uso
desta tecnologia, \textit{i.e.}, não requer que o desenvolvedor conheça o SGX
SDK, e o gerenciamento de memória de suas aplicações SGX. Nós avaliamos esta
ferramenta, por meio de experimentos, comparando o seu desempenho com o
desempenho de uma aplicação desenvolvida usando o modelo de programação
tradicional do Intel SGX. Os resultados obtidos nos experimentos indicam que
o DynSGX pode ser utilizado de forma eficiente para o processamento de funções
iterativas, gerando uma sobrecarga de apenas $2,5\%$ em relação ao uso de SGX
puro, porém o seu uso para processamento de funções recursivas pode ter
uma performance muito inferior à solução que usa SGX puro. 

\section{Limitações e trabalhos futuros}
\label{sec:conclusao_limitacoes_trabalhos_futuros}

Apesar de mostrar algumas das características de aplicações SGX que afetam o seu
desempenho, a lista aqui apresentada não é exaustiva, e não considera como duas
ou mais características em conjunto podem impactar no desempenho das aplicações.
Desta forma, mais análises precisam ser feitas para determinar outras boas
práticas a serem aplicadas no desenvolvimento de aplicações SGX.

A segunda limitação deste trabalho está na forma como foi avaliada a segurança
das aplicações, onde apenas foi considerado o tamanho do TCB.

Uma outra limitação deste trabalho diz respeito à ferramenta proposta, DynSGX.
O DynSGX foi desenvolvido com o intuito de funcionar como uma plataforma de
função como um serviço (FaaS, do inglês \textit{Function as a Service}). A sua
implementação atual, porém, requer que ambos o desenvolvedor e o usuário das
funções carregadas no \texttt{DynSGX enclave} sejam a mesma pessoa, de modo a
ganhar confiança em relação a segurança e privacidade dos dados processados.
Além disso, como citado anteriormente, a tarefa de remover funções do enclave do
DynSGX é demasiadamente mecanizada. A automatização desta tarefa deve ser
tratada em uma futura versão do DynSGX.

Como possíveis trabalhos futuros, além de solucionar as limitações destacadas,
propomos a evolução do DynSGX, de forma a transformá-lo em uma plataforma
completa de aplicações \textit{serverless}, possibilitando o desenvolvimento e
uso de aplicações por pessoas diferentes, com garantias de segurança e
privacidade de dados e código carregados na plataforma. Essa abordagem permite
maiores facilidade e segurança no desenvolvimento de aplicações SGX, bem como um
melhor gerenciamento automático de memória utilizada pelos enclaves do DynSGX.