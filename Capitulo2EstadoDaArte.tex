\chapter{Estado da arte e trabalhos relacionados}
\label{chapter:estadoarte}

Neste capítulo, mostramos e discutimos as diferentes soluções existentes
para garantia de segurança e privacidade de dados. Primeiramente exploramos os
conceitos de \textbf{TCB} e \textbf{primitivas criptográficas} e a importância
de ambos no desenvolvimento de soluções de segurança e privacidade em sistemas
computacionais. Após, são discutidas as soluções de \textbf{segurança baseadas
em \textit{software}}, e em seguida, as soluções de \textbf{segurança baseadas
em \textit{hardware}}. Mais à frente, são apresentados os trabalhos relacionados
mais relevantes à esta pesquisa. Por fim, é apresentado um breve sumário de como
este trabalho aborda o uso de uma das tecnologias atuais, com o objetivo de
tirar melhor proveito da mesma.

\section{Base de computação confiável}
\label{sec:estadoarte_tcb}

A base de computação confiável (TCB, do inglês \textit{Trusted Computing Base})
é um dos principais conceitos utilizados no desenvolvimento de aplicações
seguras. Ela foi primeiramente definida como sendo a combinação de \textit
{kernel} e processos confiáveis, \textit{i.e.}, que podem violar as regras de
controle de acesso de um sistema~\cite{rushby1981design}. Mais à frente, TCB foi
descrito como sendo uma pequena porção de \textit{software} e \textit{hardware}
na qual a segurança de um sistema computacional depende, e que se diferencia de
uma porção muito maior, que pode funcionar de forma incorreta, sem afetar a
segurança do sistema~\cite{lampson1992authentication}.

A definição mais adotada atualmente, entretanto, por ser mais formal, é a de que
o TCB de um sistema computacional é o conjunto de mecanismos de proteção
contidos no mesmo, incluindo \textit{software}, \textit{firmware} e \textit
{hardware}, que combinados são responsáveis pela aplicação de políticas de
segurança computacionais~\cite{qiu1985trusted}.

Toda aplicação segura precisa definir um TCB, e confiar nele como bloco central
para as garantias de segurança da aplicação. Em outras palavras, o TCB é a única
porção de uma aplicação que, possuindo uma vulnerabilidade, poderia ser atacada
para tentar obter controle indevido sobre a mesma.

Considerando que quanto maior uma aplicação, maior o número de \textit{bugs} e
vulnerabilidades esperado de se encontrar nela~\cite{mcconnell2004code}, uma boa
prática na arquitetura e desenvolvimento de um TCB é mantê-lo o menor possível.

\section{Primitivas criptográficas}
\label{sec:estadoarte_primitivas_cripto}

No centro de quaisquer sistemas computacionais seguros estão as primitivas
criptográficas. Elas são os blocos mais básicos de tais sistemas, e geralmente
são criados para realizar uma tarefa bem específica, e de forma bastante
confiável.

As principais tarefas que podem ser realizadas por primitivas criptográficas --
lembrando que cada primitiva deve realizar apenas uma função -- são:
\begin{description}
    \item [Função \textit{Hash}] -- Produz um valor reduzido da mensagem
    original. Geralmente é utilizado para verificar a integridade da mensagem.
    \item [Autenticação] -- Ato de verificar a identidade de alguma entidade.
    \item [Cifragem simétrica] -- Permite cifrar e decifrar dados usando uma
    mesma chave.
    \item [Cifragem assimétrica] -- Permite cifrar e decifrar dados com um par
    de chaves distintas (uma pública e uma privada).
    \item [Assinatura digital] -- Permite verificar a autoria de uma mensagem.
    \item [Geração de números (pseudo) aleatórios] -- Utilizado para prover
    aleatoriedade necessária para a segurança de várias operações.
\end{description}

Diferentes implementações de primitivas criptográficas podem prover diferentes
níveis de segurança. Podemos, dessa forma, dizer que uma primitiva provê $N$
\textit{bits} de segurança, significando que são necessárias $2^N$ operações
computacionais para quebrar a sua segurança.

Por constituir a base da construção de sistemas seguros, uma primitiva
criptográfica precisa ser bastante confiável na realização de sua funçao. Isto
é, ela precisa prover exatamente o nível de segurança prometido. Exemplificando,
se uma primitiva diz prover um nível de segurança de $X$ bits, mas pode ter sua
segurança quebrada com $X-1$ \textit{bits} (necessita apenas da metade de
operações computacionais que deveria necessitar para ser quebrada), todos os
protocolos e sistemas contruídos baseados nela passam a ser considerados
vulneráveis.

Antes de serem utilizadas, primitivas criptográficas devem ser amplamente
testadas e ter seu nível de segurança formalmente provado. Este é um processo
bastante lento. Por todos os motivos apontados, e também pela possibilidade da
inserção de erros durante o processo de desenvolvimento, recomenda-se que
desenvolvedores não tentem criar as suas próprias primitivas criptográficas, mas
utilizem as que já estão disponíveis e acolhidas pela comunidade de criptólogos
e de segurança e privacidade de dados~\cite{menezes1996handbook}.

Por fim, como dito anteriormente, primitivas criptográficas realizam uma única
função, o que, geralmente, não é suficiente para suprir todas as necessidades de
um sistema seguro. Dessa forma, é necessário combinar o uso de várias delas para
a construção de protocolos e sistemas seguros. Por exemplo, para que duas partes
se comuniquem de forma segura, somente cifrar o dado do lado do remetente e
decifrá-lo no lado do destinatário não é o suficiente. É necessário, também,
usar alguma estratégia para verificar a integridade do dado transmitido, uma vez
que um atacante no meio do caminho poderia modificar a mensagem enviada.

\section{Segurança de dados baseada em \textit{software}}
\label{sec:estadoarte_seg_soft}

Há várias estratégias para prover segurança e pravacidade através do uso de
\textit{software}, sendo a maioria delas baseada no uso de criptografia, que
foram desenvolvidas com o objetivo de prover segurança e privacidade de dados.
É importante citar que o uso de técnicas de segurança de dados baseadas em
\textit{software} implica na adição de bibliotecas e APIs ao TCB de uma
aplicação, tornando-a mais suscetível a vulnerabilidades inseridas durante o
processo de desenvolvimento das mesmas.
Além das estratégias baseadas no uso de criptografia, podemos destacar também a
criação de ambientes isolados, como máquinas virtuais, e cônteineres. Detalhamos
cada uma destas estratégias a seguir.

\subsection{Sistemas criptográficos}
\label{subsec:estadoarte_segsoft_sistemas_cripto}

Sistemas criptográficos são constituídos por um conjunto de primitivas
criptográficas, geralmente sendo compostos de uma primitiva para geração de
chaves, uma primitiva para cifrar dados, e uma primitiva para decifrar dados.

Um sistema criptográfico pode ser definido formalmente como a tupla ({\ttfamily
P,C,K,$\varepsilon$,D}), onde:
\begin{itemize}
    \item {\ttfamily P} é o conjunto de \textbf{textos puros} existentes;
    \item {\ttfamily C} é o conjunto de \textbf{cifrotextos} existentes;
    \item {\ttfamily K} é o conjunto de \textbf{chaves} existentes;
    \item {\ttfamily $\varepsilon$} é o conjunto de funções $E_{k}: P
    \rightarrow C$, com $k \in K$, que cifra um texto puro em um cifrotexto;
    \item {\ttfamily D} é o conjunto de funções $D_{k}: C
    \rightarrow P$, com $k \in K$, que decifra um cifrotexto em um texto puro;
    \item $\forall e \in K, \exists d \in K | D_{d}(E_{e}(p)) = p, \forall p \in
    P$.
\end{itemize}

Em relação à última propriedade citada, é importante apontar que existem
sistemas criptográficos simétricos e assimétricos. Em um sistema criptográfico
simétrico, temos que $e = d$, \textit{i.e.}, a chave utilizada para cifrar um
texto puro é a mesma utilizada para decifrar o cifrotexto gerado. Como exemplo
de algoritmo de criptografia simétrica atualmente em uso, podemos citar o AES,
do inglês \textit{Advanced Encryption Standard}~\cite{aes2001}.

Já em um sistema criptográfico assimétrico, também conhecido como criptografia
de chave pública, temos que $e \neq d$, \textit{i.e.}, há a necessidade da
criação de duas chaves distintas, sendo uma mantida em segredo (\textbf{chave
privada}), e uma tornada pública (\textbf{chave pública}) para a utilização por
outras partes. Se uma das chaves é utilizada para cifrar um texto plano, o seu
par deve ser utilizado para decifrar o cifrotexto. Além das funcionalidades de
cifrar e decifrar textos, sistemas criptográficos assimétricos também podem ser
utilizados para provar a origem de uma mensagem (\textbf{assinatura digital}),
negociar uma chave simétrica de forma segura (\textit{e.g.}, algoritmo 
Diffie-Hellman~\cite{merkle1978secure}), e até mesmo para realizar processamento
de dados utilizando apenas seu cifrotexto (\textit{e.g.}, criptografia
homomórfica). Algoritmos de criptografia assimétrica, entretanto, demandam um
processamento geralmente muito mais complexo do que algoritmos de criptografia
simétricos. É interessante, portanto, combinar ambas as técnicas para alcançar
os objetivos de segurança sem comprometer a performance do sistema.

A técnica de criptografia homomórfica, citada anteriormente, permite o
processamento de dados na forma de cifrotexto. Isso possibilita, por exemplo,
que seja realizada a busca por uma entrada cifrada em um banco de dados, também
cifrado, sem permitir que nenhuma informação sobre a entrada e sobre o banco de
dados seja extraída por algum atacante em controle da máquina onde a operação
está sendo executada. Estudos anteriores~\cite{sac2017}, porém, mostraram que
esta técnica é extremamente custosa em comparação a outras técnicas de segurança
baseadas em \textit{hardware}, sendo impraticável o seu uso em aplicações reais.

\subsection{Isolamento de recursos}
\label{subsec:estadoarte_segsoft_isolamento_recursos}

Outra forma de proteção de dados baseada em \textit{software}, é o uso de
técnicas de isolamento de recursos de uma máquina física, para que sejam
utilizados, idealmente, por um único usuário ou aplicação.

Uma das técnicas de isolamento de recursos, amplamente utilizada na atualidade,
é a de virtualização. Na técnica de virtualização, ambientes isolados, aqui
chamados de \textbf{máquinas virtuais} (VMs, do inglês \textit{Virtual
Machines}), são criados com o objetivo de emular as funcionalidades de uma
máquina física, e proteger estes ambientes contra outros que coexistam na mesma
máquina hospedeira. Várias implementações de virtualização existem hoje no
mercado~\cite{qemu,xen,kvm,vmware}, e são utilizadas principalmente em
ambientes de computação na nuvem, sejam elas privadas ou públicas.

Para a criação de VMs, é necessário o uso de um componente conhecido como
hipervisor. Ele é responsável por criar e executar as VMs, assim como prover
uma interface entre as VMs e o \textit{hardware} real. Desta forma, o uso de
virtualização para a proteção de dados sensíveis requer a adição deste
componente ao TCB da aplicação como um todo.

Ao decorrer do tempo, várias vulnerabilidades foram encontradas em hipervisores,
possibilitando que atacantes, a partir de VMs, ganhassem total controle sobre o
hipervisor, o utilizassem para ganhar acesso à outras VMs residentes na mesma
máquina~\cite{kortchinsky2009cloudburst}.

\subsection{Considerações}
Concluímos a seção apontando que soluções de segurança baseadas em \textit
{software} podem gerar um grande TCB nas aplicações que as utilizam, e
consequentemente, podem aumentar a probabilidade de inserção de uma série de
vulnerabilidades na aplicação. Além disso, outras soluções, apesar de já terem
seu uso estabelecido na comunidade, como é o caso de criptografia assimétrica,
ou estarem sendo bastante pesquisadas, como é o caso de criptografia
homomórfica, acabam gerando um grande custo computacional, prejudicando o
desempenho das aplicações que as utilizam.

\section{Segurança de dados baseada em \textit{hardware}}
\label{sec:estadoarte_seg_hard}

Como introduzido na Seção~\ref{sec:intro_contextualizacao}, o crescente volume
de dados sensíveis sendo enviados e processados em ambientes de computação na
nuvem levou à definição de um TEE, que facilitaria o desenvolvimento de soluções
com garantias de segurança e privacidade de dados.

As soluções para segurança de dados baseadas em \textit{hardware} geralmente se
dão através da implementação de um TEE como parte de um microprocessador, com o
objetivo de isolar dados de um processo em execução dos demais processos que
executam na mesma máquina. Tais soluções, em geral, são computacionalemente mais
eficientes que soluções baseadas em \textit{software}, porém, para serem
utilizadas, elas requerem o uso de \textit{hardware} específico, que nem sempre
está disponível nos ambientes onde as apicações serão executadas.

Como exemplos de implementações existentes no mercado, podemos citar as soluções
que buscam separar o espaço de memória de uma máquina em duas áreas distintas,
uma segura, e uma sem garantias de segurança, como é o caso do ARM
TrustZone~\cite{trustzone} e o AMD SME/SEV~\cite{amdSMESEV}. Nestas
implementações, desenvolvedores de aplicações definem quais porções de dados
devem ser considerados sensíveis, e o processador se encarrega de mantê-los em
memória de uma forma criptograficamente segura. Em ambos os casos, todos os
dados sensíveis, mesmo que pertencentes a diferentes aplicações são mantidos em
um mesmo TEE. Uma outra implementação possível, que é a adotada pelo Intel
SGX~\cite{mckeen2013innovative}, busca separar o espaço de memória de uma
máquina entre uma área insegura, e várias áreas isoladas seguras. Nesta solução,
cada aplicação pode possuir uma área de memória segura para armazenar e
processar dados sensíveis, e duas aplicações distintas não podem compartilhar o
mesmo espaço de memória seguro, alcançando total isolamento entre aplicações
distintas.

Dentre as tecnologias para segurança baseada em \textit{hardware} mencionadas,
Intel SGX é a que se mostra como sendo mais promissora, e como tendo o maior
número de pessoas envolvidas em pesquisas sobre a mesma, possivelmente devido
à facilidade de acesso a processadores compatíveis com a tecnologia. Desta
forma, SGX foi escolhida como objeto de estudo deste trabalho.

\section{Trabalhos relacionados}
\label{sec:estadoarte_trabalhos_relacionados}

Desde a publicação inicial~\cite{mckeen2013innovative}, feita em 2013, até a
atualidade, diversas pesquisas sobre Intel SGX já foram divulgadas. Estas
pesquisas são voltadas para diversos aspectos da tecnologia, que incluem seu uso
em aplicações reais~\cite{krawiecka2016protecting,fetzer2016building,sac2017},
ferramentas para facilitar o desenvolvimento de aplicações SGX~\cite
{scone,tsai2017graphene,baumann2015shielding,DynSGXCloudCom2017}, implementações
de ataques de canal lateral~\cite
{hahnel2017high,cachezoom2017,schwarz2017malware}, e construção de enclaves mais
seguros~\cite{gruss2017strong,lind2017glamdring,kuvaiskii2017sgxbounds}.

Arnautov \textit{et al.} propuseram a ferramenta SCONE, que tem como objetivo
executar aplicações não modificadas dentro de enclaves SGX~\cite{scone}. Esta abordagem
facilita a criação de aplicações seguras a partir de aplicações existentes,
porém, ao inserir a aplicação por completo no enclave SGX, o seu TCB é
aumentado. Esta solução vai de encontro com a recomendação da Intel, de que
aplicações SGX devem manter apenas o mínimo de código possível dentro dos
enclaves. Por outro lado, ela facilita o processo de criação de uma aplicação
segura, pois não exige que o desenvolvedor aprenda a usar toda a API do Intel
SGX para utilizar suas capacidades de segurança e privacidade.

Lind \textit{et al.} propuseram a ferramenta Glamdring, que tem como objetivo
facilitar o desenvolvimento de aplicações SGX, provendo o particionamento
automático das mesmas~\cite{lind2017glamdring}. Para isso, o desenvolvedor deve
utilizar anotações de código para marcar quais dados são considerados sensíveis
na sua aplicação. Após a marcação feita manualmente pelo desenvolvedor, a
ferramenta gera o código da aplicação particionado entre as partes segura e
insegura, mantendo o mínimo possível de código dentro do enclave. Como veremos
mais à frente, no Capítulo~\ref{chapter:abordagem} esta estratégia não é a ideal
em todos os cenários.

Por fim, é importante falar sobre os trabalhos sobre ataques de canal lateral. A
Intel deixa claro na documentação do SGX~\cite{sgxdeveloperreference} que a
tecnologia, por si só, não é protegida contra ataques de canal lateral. Isto
quer dizer que é possível deduzir informações sobre os dados sensíveis sendo
processados através da observação do comportamento de outros componentes do
sistema, como comportamento da \textit{cache}, utilização de \textit{CPU},
consumo de energia, \textit{etc}. Götzfried \textit{et al.} demonstraram ser
possível realizar ataques à \textit{cache} do processador para obter dados
sensíveis do SGX~\cite{gotzfried2017cache}.

Apesar de ataques de canal lateral serem possíveis de se realizar em aplicações
SGX, eles podem ser evitados através de boas práticas no desenvolvimento das
aplicações seguras. Por exemplo, Gruss \textit{et al.} propuseram um mecanismo
capaz de evitar o mesmo tipo de ataques à \textit{cache} do processador citado
anteriormente~\cite{gruss2017strong}.

\section{Considerações}
\label{sec:estadoarte_consideracoes_finais}

Devido ao crescente volume de dados sensíveis sendo enviados e processados em
ambientes de computação na nuvem, faz-se necessário tomar medidas para prover
proteções contra publicação e modificação de tais dados de forma indevida. Neste
capítulo discutimos que tais medidas podem ser implementadas através do uso de
técnicas baseadas em \textit{hardware} ou \textit{software}.

As soluções baseadas em \textit{software} têm a vantagem de não depender de um
\textit{hardware} específico para serem utilizadas, porém, na maioria das vezes,
incorrem no aumento do TCB das aplicações, e na perda de desempenho das mesmas.

As soluções baseadas em \textit{hardware}, por sua vez, em geral têm um menor
TCB, diminuindo o risco da adição de vulnerabilidades durante o processo de
desenvolvimento das aplicações, e têm uma melhor performance em comparação a
soluções baseadas em \textit{software}, porém, necessitam ser executadas em
ambientes com uma configuração de \textit{hardware} específica disponível.

Uma das soluções baseadas em \textit{hardware} é o Intel SGX. Ela pode ser usada
para proteger código de aplicações contra modificação, e proteger dados de
aplicações contra modificação e acessos não autorizados. Várias pesquisas sobre
aplicações de Intel SGX, suas possíveis vulnerabilidades, e como mitigar estas
vulnerabilidades, podem ser encontradas na literatura, porém elas não avaliam o
impacto de vários aspectos na performance e nas garantias de segurança de tais
aplicações.

Considerando todos estes pontos, nos capítulos que seguem, detalhamos a
tecnologia Intel SGX, apontamos como diversos fatores devem ser considerados
durante o processo de desenvolvimento de aplicações seguras usando SGX, e
propomos um conjunto de ferramentas que facilitam o desenvolvimento de
aplicações SGX, bem como ajudam a superar alguns desafios do uso da tecnologia
em ambientes reais.