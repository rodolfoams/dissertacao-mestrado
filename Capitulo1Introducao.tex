\chapter{Introdução}
\label{chapter:intro}

\section{Contextualização}
\label{sec:intro_contextualizacao}

\subsection{Segurança e privacidade}
\label{subsec:intro_contextualizacao_seguranca_privacidade}

Vivemos em um mundo que está a cada dia mais conectado, onde pessoas estão
constantemente enviando dados pessoais para ambientes que não podem ser
controlados por elas. Ocasionalmente, esses dados são destinados a tornarem-se
públicos. Na maioria dos casos, porém, eles devem ser mantidos privados e/ou
seguros -- tais dados serão doravante denominados \textbf{dados sensíveis}.
Nesse sentido, desenvolvedores de aplicações devem encontrar formas de proteger
os dados de seus usuários contra roubo ou modificações indevidas a todo custo.

Além disso, desenvolvedores e empresas frequentemente hospedam suas aplicações
em ambientes de nuvem pública, com o objetivo de aumentar sua disponibilidade e
escalabilidade, ou até mesmo para diminuir os custos com infraestrutura física
\cite{armbrust2010view}. Em tais ambientes, múltiplas aplicações de diferentes
donos podem residir no mesmo servidor físico, possibilitando que usuários
maliciosos explorem vulnerabilidades de segurança existentes para obter dados
sensíveis de outros usuários.

Para  entender os requisitos de aplicações que lidam com dados sensíveis em
cenários como os supracitados, precisamos primeiramente definir os conceitos de
segurança e privacidade. Segurança diz respeito a proteções contra acesso,
modificação e remoção de dados sensíveis por pessoas não autorizadas.
Privacidade, por sua vez, diz repeito à proteção contra o uso de dados
sensíveis para propósitos não autorizados.

Considerando esses conceitos, aplicações que lidam com dados sensíveis devem
considerar os seguintes requisitos:

\begin{itemize}
    \item Confidencialidade -- impedir o acesso a dados sensíveis por pessoas
    não autorizadas;
    \item Disponibilidade -- garantir que dados sensíveis possam ser recuperados
    por pessoas autorizadas sempre que requisitados;
    \item Integridade -- impedir a modificação de dados sensíveis por pessoas
    não autorizadas;
    \item Privacidade -- impedir o uso de dados sensíveis para propósitos não
    autorizados.
\end{itemize}

Há situações em que desenvolvedores também desejam manter o código de suas
aplicações privado e seguro, seja pelo uso de código proprietário ou por algum
outro motivo qualquer. Nessas situações, o código de suas aplicações devem ser
tratados como dados sensíveis.

\subsection{Ambiente de execução confiável}
\label{subsec:intro_contextualizacao_tee}

O crescente volume de dados sensíveis sendo enviados, armazenados e processados
em ambientes de computação na nuvem, por um número cada vez maior de
dispositivos móveis, e as demandas de segurança e privacidade de tais dados
levaram um fórum de operadores de redes móveis (OMTP, do inglês \textit{Open
Mobile Terminal Platform}) a definir um padrão para um ambiente de execução
confiável (TEE, do inglês \textit{Trusted Execution Environment}).

Inicialmente, um TEE foi definido pela OMTP como um conjunto de componentes de
\textit{hardware} e \textit{software} que facilitam o suporte de aplicações que
satisfaçam os requisitos de um dos dois níveis de segurança definidos pela
mesma. O primeiro nível de segurança prevê proteção apenas contra ataques de
\textit{software}, enquanto o segundo nível de segurança prevê proteção contra
ataques de \textit{hardware} e de \textit{software}~\cite{omtpadvanced2009}.

Atualmente, o padrão TEE é definido pela \textit{GlobalPlatform}, uma associação
industrial que visa desenvolver padrões para processamento seguro e confiável
em serviços digitais e outros dispositivos. A \textit{GlobalPlatform} define TEE
como uma área segura do processador principal em um \textit{smart phone} ou
qualquer dispositivo conectado a ele. O TEE garante que dados sensíveis são
armazenados, processados e protegidos em um ambiente isolado e confiável~\cite
{globalplatform2017}.

De forma a facilitar o desenvolvimento de serviços que garantam segurança e
privacidade de dados sensíveis, várias implementações de TEE foram feitas~\cite
{omtpadvanced2009,logictrusted,trustonic2017,mccune2008flicker,amdSMESEV}, entre
as quais destacamos o Intel SGX, do inglês \textit{Software Guard eXtensions},
objeto de estudo deste trabalho de mestrado.

\subsection{Intel SGX}
\label{subsec:intro_contextualizacao_intel_sgx}

Com o passar do tempo, não só dispositivos móveis necessitavam de um TEE, mas
sim todo um nicho de dispositivos que lidam com dados sensíveis. Um exemplo
desses dispositivos são os computadores, sejam eles servidores usados em
ambientes de computação na nuvem, sejam eles máquinas usadas por clientes.
Visando este tipo de dispositivos, em 2013 a Intel definiu uma implementação
própria de TEE conhecida como SGX.

%%%TODO: eu preferiria mencionar por extenso dentro da seção que fala sobre ele.

Intel SGX é um conjunto de instruções incorporado às arquiteturas Intel 64 e
IA-32 que permite a criação de contêineres baseados em \textit{hardware},
chamados \textbf{enclaves}. Dados sensíveis que sejam carregados em um enclave são
protegidos contra modificação ou acesso não autorizado; Código de aplicações que
executem dentro de enclaves SGX são protegidos contra modificação~\cite
{mckeen2013innovative}.

Intel SGX está disponível em processadores da 6\textsuperscript{a} geração da
família Intel Core i, com arquitetura \textit{Skylake}, ou processadores mais
novos da mesma família, bem como em processadores da família Xeon, a partir da 5\textsuperscript{a} versão. Tais processadores estão disponíveis no mercado desde o segundo semestre do ano de  2015.

Atualmente muitas pesquisas vêm sendo desenvolvidas e publicadas\footnote{\url{https://software.intel.com/en-us/sgx/academic-research}} sobre a
aplicação de Intel SGX no mundo real~\cite{hoekstra2013using}, bem como
vulnerabilidades encontradas na tecnologia, e ações necessárias para mitigar estas vulnerabilidades~\cite{swami2017intel}.

Por ser uma tecnologia bastante promissora e já estar presente em um grande
número de processadores atualmente em uso, iremos utilizá-la como objeto de
estudo neste trabalho de dissertação.

\section{Problema investigado}
\label{sec:intro_problema_investigado}

No ano de 2016, estimou-se que o custo anual com crimes cibernéticos em alguns
países chega a um total de $USD\ 74$ milhões (setenta e quatro milhões de
dólares americanos)~\cite{costcybercrime2016}. Além disso, com a crescente produção de
dados sensíveis enviados para ambientes de computação na nuvem, cresce também a
quantidade de ataques cibernéticos com o intuito de obter e usar tais dados de
forma não autorizada. Levando em consideração esses fatos, desenvolvedores de
aplicações que lidam com dados sensíveis devem cada vez mais procurar formas de
garantir a segurança e a privacidade dos dados de seus clientes.

Uma forma de proteger os dados de usuários é através do uso de um TEE. Uma das
implementações de um TEE, que está cada vez mais difundida em máquinas de
usuários e em máquinas servidoras de computação na nuvem, é o Intel SGX. Assim
como qualquer outra tecnologia, Intel SGX também possui vulnerabilidades, como
falta de proteção contra ataques de canal lateral, que podem impactar as
garantias de segurança providas por aplicações que façam uso da tecnologia, bem
como limitações, como um pequeno espaço de memória disponível, que podem
impactar o desempenho dessas aplicações devido ao seu mau uso.

Para tirar o melhor proveito de Intel SGX, desenvolvedores devem tomar várias
decisões na hora de desenvolver suas aplicações, como: $(i)$ qual porção da
aplicação realmente precisa ser protegido pelas instruções do SGX, $(ii)$ como
gerenciar a memória consumida por suas aplicações, e $(iii)$ como evitar ataques
de canal lateral.

\section{Contribuições e relevância}
\label{sec:intro_contribuicoes}

Este trabalho tem como objetivo abordar os problemas de vulnerabilidades e baixa
eficiência de aplicações, oriundos de más práticas no uso de Intel SGX,
investigando arquitetura, implementação e implantação de aplicações que façam
uso dessa tecnologia. Com esse objetivo, são propostas várias práticas que devem
ser consideradas durante o processo de desenvolvimento de aplicações que lidam
com dados sensíveis. Tais práticas levam em consideração as vulnerabilidades e
limitações do Intel SGX, bem como o ambiente onde essas aplicações serão
executadas. Considerando essas práticas, desenvolvedores podem tomar decisões
como: $(i)$ como dividir suas aplicações no modelo de programação do Intel SGX;
$(ii)$ como particionar os dados a serem processados por suas aplicações;
$(iii)$ como gerenciar a memória consumida por suas aplicações; $(iv)$ como
garantir privacidade do código de suas aplicações.

Estas decisões tomadas pelo desenvolvedor trazem benefícios como: $(i)$ maior
proteção de dados sensíveis de clientes; $(ii)$ diminuição da sobrecarga gerada
pelo uso de Intel SGX; $(iii)$ diminuição na sobrecarga gerada pelo consumo de
memória excedente à disponível; $(iv)$ proteção de código proprietário.

Este trabalho também introduz a ferramenta DynSGX~\cite{DynSGXCloudCom2017},
desenvolvida ao longo da pesquisa do mestrado, que ajuda desenvolvedores a
proteger o código de suas aplicações usando enclaves SGX, bem como possibilita
uma melhor gerência de memória de enclaves ocupada por código das aplicações.

O trabalho é relevante por tratar de uma tecnologia que está se tornando padrão
em processadores tanto de uso pessoal, quanto de servidores, e pela ausência de
guias de desenvolvimento de aplicações SGX que indiquem as implicações de
decisões tomadas neste processo.

\section{Metodologia}
\label{sec:intro_metodologia}

A partir de levantamento bibliográfico do estado da arte, que serviu de base
para investigações sobre limitações e vulnerabilidades do uso de Intel SGX para
proteção de dados sigilosos, foi produzida uma lista de aspectos que devem ser
considerados por programadores ao desenvolver uma aplicação SGX. Baseado nessa
lista, foram desenvolvidos experimentos para avaliação do impacto em segurança e
performance causado por cada um desses aspectos. Em seguida, os experimentos
foram implementados para avaliação, considerando diferentes abordagens para cada
um dos aspectos listados, que serão detalhadas no Capítulo~\ref
{chapter:abordagem}. A implementação de todos os componentes, bem como da
comunicação entre diferentes componentes, foi feita usando as linguagens C e
C++.

Após a implementação, os experimentos foram executados em uma máquina que possui
SGX habilitado, utilizando configurações comuns em soluções existentes no
mercado e na literatura. Nesta máquina, foram comparadas diferentes abordagens
usadas para os aspectos estudados. Essa comparação foi feita coletando métricas
como: $(i)$ consumo de memória, $(ii)$ tempo de execução de tarefas, $(iii)$
tamanho do \textit{Trusted Computing Base} de uma aplicação, com implicações em
segurança de dados, e $(iv)$ garantias de segurança de código de aplicações.

Por fim, uma nova ferramenta foi proposta, com o objetivo de facilitar o
desenvolvimento de aplicações que façam uso do SGX, levando em consideração as
dificuldades e boas práticas discutidas neste trabalho.

\section{Organização da dissertação}
\label{sec:intro_organizacao}

Este trabalho de dissertação é organizado da seguinte forma:
\begin{itemize}
    \item O Capítulo~\ref{chapter:estadoarte} mostra as principais soluções para
    segurança de dados, sejam elas baseadas em \textit{software} ou em \textit
    {hardware}, bem como trabalhos relacionados à pesquisa apresentada nesta
    dissertação;
    \item No Capítulo~\ref{chapter:sgx} é detalhada a tecnologia Intel SGX,
    incluindo uma descrição geral da tecnologia, suas vantagens, modelo de
    programação, modelo de memória, limitações e vulnerabilidades no seu uso;
    \item Em seguida, no Capítulo~\ref{chapter:abordagem} é detalhada a
    abordagem aqui proposta, onde descrevemos cada um dos aspectos analisados no
    desenvolvimento de aplicações seguras, bem como os experimentos executados
    para avaliação de boas práticas considerando cada um desses aspectos, e os
    resultados obtidos nesta avaliação;
    \item No Capítulo~\ref{chapter:dynsgx} introduzimos a ferramenta DynSGX,
    criada com o objetivo de facilitar o desenvolvimento de aplicações SGX, e adicionar
    garantias e funcionalidades não providas pela tecnologia Intel SGX isoladamente.
    \item Por fim, no Capítulo~\ref{chapter:conclusao} é apresentada a conclusão
    do trabalho, sumariando os resultados e listando limitações e trabalhos
    futuros.
\end{itemize}